\section{Motivation}

Epidemiological studies have recently pointed to the potential of airborne
transmission beyond the close physical contact distance of 1--2 meters
\cite{prather2020airborne}. While the overall contribution of this transmission
route remains unknown, multiple studies have shown that clusters -- i.e., events
with a comparatively high number of transmission events -- play a crucial role
in the epidemiological dynamics of the COVID-19 pandemic
\cite{adam2020clustering, wong2020evidence}.

Such clusters seem to predominantly occur in crowded indoor environments with limited ventilation. Such environments may include restaurants, bars, classrooms, and places of worship as well as events such as concerts, meetings, and private reunions. For quarantine considerations, contact tracers are thus increasingly including not only individuals who were in close proximity to an infected person, but also those who were present at the same location. 

These places and events can host a considerable amount of people, and notifying
them thus increases the load of contact tracers. When this load is already high due to the amount of positive cases, manual notification associated to presence can become a bottleneck. Digital presence tracing solutions aim at reducing the load of contact tracers in two dimensions: supporting contact tracers in deciding who to notify, and facilitating the notification process. 

The deployment of digital presence solutions requires the creation of infrastructure that inevitably generates new data. The potential misuse of this data raises unease in the population and put at risk the wide adoption of these technologies which in turn reduces their utility as support for contact tracing.

For these reasons, it is important to design digital presence tracing in a privacy-preserving manner, anticipating the dangers associated with mass deployment, to facilitate adoption. In this work, we take into account the following four factors:

\paragraph{Checks on power and population control}
Digital presence tracing systems facilitate the notification of individuals who meet epidemiologically-defined criteria (e.g., were present at a location within a certain time window compared to a potential index case). Deployed presence tracing systems do so at the sole discretion of a central authority such as a public health agency. Such infrastructure can, however, be easily abused if a central authority utilises it for non-epidemiological purposes: With a registry of locations, and background knowledge (e.g., about the demographics or characteristics of visitors to those locations), segments of the population can be ordered to quarantine with no additional infrastructure or cost.

This contrasts with proximity tracing technologies such as Exposure Notification, where to trigger notifications with this effect would require the deployment of infrastructure (e.g., Bluetooth emitters, or coopted smartphone applications~\cite{DehayeR20}) which could not operate with the same certainty, costless scalability, or undetectability. When combined with legal obligations to quarantine and/or individuals' personal desire not to infect others, systems of presence tracing that operate at the discretion of a central authority could serve as a method of population control. It could, for example, be used as a tool for voter suppression, particularly as political actors often already hold detailed background knowledge as to the location and demographics of voters holding different electoral intentions. As a consequence, digital presence tracking systems should be designed to effectively meet genuine epidemiological goals while introducing safeguards against this potential abuse of power by central authorities.

\paragraph{Safeguards for user privacy}
A second group of concerns around presence tracing relates to issues of individual privacy. Systems which leak individuals’ location histories can be abused in a variety of ways depending on to whom the leak occurs. Leaks to central authorities may result in persecution; leaks to other users, such as perpetrators of intimate partner violence and abuse, may exacerbate coercive control~\cite{ChatterjeeDOHPF18,FreedPMLRD18,LevyS20}. Digital presence tracing systems should be designed to remove or minimise these risks.

\paragraph{Concerns for the confidentiality of locations}
A third group of concerns relates to the confidentiality of locations, and societal harms that can result from rendering the existence of social gatherings, meetings and communities centrally legible and machine-readable. While many venues already volunteer or are legally compelled to be part of large and often publicly accessible databases (e.g., bars for licensing or mapping purposes), this is not the case for all epidemiologically relevant gatherings. For some, such as political or religious gatherings, a technology that demands they formally register their existence to an entity outside their community (such as a central authority) may itself pose a threat regardless of the individual privacy protections or safeguards on the power to issue notifications\footnote{See e.g., in the United Kingdom, The Health Protection (Coronavirus, Collection of Contact Details etc and Related Requirements) Regulations 2020 schedules 1--4.}. While technological safeguards may help mitigate some concerns, at the heart of this issue is the definition (which may be in law) of what constitutes a relevant `location' for epidemiological purposes. Choosing to broaden this beyond existing, self-published locations (e.g., most restaurants, bars, cafés) will be a political choice that cannot be easily specified or limited by a technological protocol, and the breadth of the definition should be considered in the proportionality of introducing any presence tracing system. Therefore, to ensure that the protection provided by the system is independent of policy decisions, we aim to build a solution that does not rely on such a registry of locations.

Additionally, the system should aim to maintain confidentiality of which locations were visited by SARS-CoV-2-positive people to prevent stigmatization and shaming of owners or organisers, and visitors.

\paragraph{Gracious dismantling}
A final concern relates to the sunset of the system. Because the system is introduced in a situation of emergency, it is important to ensure that, after this situation ends, the impact of the system is minimal. Ideally, control on the sunset should be on the users' side, without creating a dependency on authorities. Successful sunset of the system should not rely on authorities dismantling infrastructure (e.g., by stopping servers, or deleting databases).

Moreover, it is desirable that once the system is not necessary anymore, authorities dismantle all infrastructure supporting the system. In other words, none of the infrastructure created to support the system should prevail after system operation stops.
