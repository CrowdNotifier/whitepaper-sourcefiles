\section*{Executive Summary}
We are publishing this document to inform the discussion revolving around the design and implementation of proximity tracing systems. We seek feedback from a broad audience on the high-level design, its security and privacy properties; so that missing requirements and protections can be identified and the overall system can be improved.

Contract tracing aims to slow down the spread of viruses (e.g., SARS-CoV-2) by identifying contacts of persons that have been infected with the virus. Traditionally this is a manual effort where contact tracers interview SARS-CoV-2-positive people to determine who their contacts are. Digital proximity tracing aims to augment this effort by rapidly notifying contacts that have been in close proximity to an infected person, even if the infected person could themselves not identify that contact.

Epidemiologists have pointed out the need for presence tracing to notify people that have been sharing a semi-public location with a SARS-CoV-2-positive person for a prolonged time. Such locations include restaurants, bars, classrooms, and places of worship; but also events such as concerts and meetings. In contrast, digital proximity tracing, such as Exposure Notification, does not identify such contacts unless the user was in close physical proximity to the SARS-CoV-2-positive person.

Existing presence tracing systems typically require a user to “check-in” to a location, such as using a hand-written or electronic form, or using a sensor on a device, such as a camera to scan a QR code. 

In this document we present a simple presence tracing system, CrowdNotifier,
that is \emph{easy to use}, will \emph{effectively notify} any presence contacts of SARS-CoV-2-positive persons, while at the same time providing \emph{strong privacy and abuse-prevention properties}.