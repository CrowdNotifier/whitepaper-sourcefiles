\section*{Executive Summary}
We are publishing this document to inform the discussion revolving around the design and implementation of presence tracing systems and to present our version, CrowdNotifier. We welcome feedback on the requirements and properties of presence tracing systems we identity as well as the design we propose, so that missing requirements and protections can be identified and the overall system can be improved.

Contract tracing aims to slow down the spread of viruses (e.g., SARS-CoV-2) by identifying close contacts of persons that have been infected with the virus. Traditionally this is a manual effort where contact tracers interview SARS-CoV-2-positive people to determine who their close contacts are. Digital proximity tracing aims to augment this effort by rapidly notifying contacts that have been in close proximity to an infected person, even if the infected person could themselves not identify that contact.

To account for airborne transmission in spaces with bad ventilation or physical but non-prolonged contacts, epidemiologists have pointed out the need to also consider people that have been sharing a semi-public location with a SARS-CoV-2-positive person for a prolonged time as contacts to be notified. Such locations include restaurants, bars, classrooms, and places of worship; but also events such as concerts and meetings. 

Contact tracing systems, therefore, need means to notify these ``presence contacts'', i.e., people present a such locations. Digital proximity tracing, such as Exposure Notification, does not provide such means, as its operation means that only people at the venue who were in close physical proximity to the SARS-CoV-2-positive person would be notified.

Existing systems to enable presence tracing typically require a user to “check-in” to a location, such as using a hand-written or electronic form, or using a sensor on a device, such as a camera to scan a QR code. 

In this document we present a simple presence tracing system, CrowdNotifier,
that is \emph{easy to use}, can be used to \emph{effectively notify} any presence contacts of SARS-CoV-2-positive persons, while at the same time providing \emph{strong privacy and abuse-prevention properties}.