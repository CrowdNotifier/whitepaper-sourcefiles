\section{Design}
The proposed system consists of two parts: (1) a fall-back paper solution and (2) a digital solution using a smartphone application that scans QR codes. The existence of the fallback solution ensures compliance with F1. Both parts are decentralized, but the digital solution achieves better privacy properties.

\subsection{Security and Privacy by Design}
\label{sec:sec-priv-design}

We first discuss how the design choices behind \name result in some of the requirements in Section~\ref{sec:security-privacy-requirements} being achieved, or being impossible to achieve.

\parait{On-device Matching} In \name all information is stored on the user devices, which are in charge of matching visited locations to trace locations and notify the user. Thus, by design, user data is neither stored centrally nor at the location (achieving PU1 and PU2).

\parait{Local generation of QR codes} Location owners generate their QR codes
locally, without connecting to a server. Thus, the system does not require a
central database of locations, achieving PL3.
We note that in the design above, PL4 is not achieved, but these uploads could be hidden by deploying dummy uploads.

\parait{(Semi-)Static QR codes for venues} Locations can use static QR codes, ensuring easy deployment for locations (fulfilling NF2). As a result of this choice -- regardless of the underlying cryptography -- PL2 becomes impossible to achieve. Users that visited a location at any time know the same information as contemporary visitors, and can thus later confirm that this location was marked as a trace location; even if the user was not there on the same day/time.

If protecting against this attack is essential, locations can instead generate new QR codes for every day or time-slot (for example, displaying them on screens). This requires the adversary to have been present at the same time as the SARS-CoV-2-positive user to determine whether the location is notified. We note that this is the best protection any system can provide as, if the adversary shared space with the SARS-CoV-2-positive user, the adversary should be notified (to fulfill F2).

\parait{Single-purpose: notify presence contacts} \name does not aim to aid contact tracers in their interview with a SARS-CoV-2-positive user. Thus, \name does not have to retain a readable list of visited locations (fulfilling PU3). Instead---as is currently the norm---contact tracers must rely on interviews with the SARS-CoV-2-positive person to determine which locations that person visited, when and for how long.


The system also does not facilitate contacting or identifying the owner of a location (e.g., a venue owner, or event organizer). Instead---as is currently the norm---contact tracers should rely on public records and information provided by the SARS-CoV-2-positive user to get in touch with the location owner.

\parait{No network connections required when entering a location/event} All necessary information is embedded in the QR code. Thus the system works even when there is no Internet connection (supporting NF1 and NF2). Moreover, the system does not generate any observable network traffic when entering locations as a result (fulfilling PU4, PU5).

\parait{No uploads by users} Notified and SARS-CoV-2-positive users do not generate any network traffic or send data to central servers, achieving PU4 and PU5.

\parait{Regular polling by phones rather than push messages} The use of push notification services such as Firebase requires that users enroll with and connect to 3rd party services (violating PU1). This is not necessary to notify contacts (F2).

\parait{User-centric dismantling} From a technical perspective, the system also supports dismantling without the collaboration of any authority by design. When users do not scan or owners do not reveal tracing keys, the system ceases its operation, fulfilling S3. However, we note that even though technically this is the case, in practice an authority could legally mandate venues to have the system in place and require scanning from users, effectively keeping the system in place. Even in this case, as the triggering of notifications cannot be monitored and reactions to notifications cannot be enforced, owners and users could still sabotage the system: owners could provide fake tracing keys that do not result in notifications, and users could ignore the notifications.

\parait{Not using the Google/Apple Exposure Notification (GAEN) framework} To use the GAEN framework, locations must operate infrastructure (violating NF2). In particular, locations would need to operate one or more Bluetooth beacons that transmit GAEN Bluetooth beacons for the location to be registered as a contact. Moreover, these beacons would need to be crafted to ensure that any mobile device inside the location would consider the location as a close contact, while avoiding false alarms to others that are not in the same place (e.g. others in a flat upstairs, or a neighbouring establishment).

\parait{Not using (GPS) location} We also decided against the use of (GPS) location to automatically determine which locations a user visits, or to determine when a user leaves such a location. First, the perceived increase in usability -- the user might not need to check in or out anymore -- is mitigated by the lack of accuracy. Indoor GPS localization is not reliable enough to distinguish between different floors and rooms; thus requiring user intervention still. Second, the use of location data to automatically check in or out requires continuous access to the user's location and the corresponding fine-grained location permission. Following what happened with contact-tracing apps, we expect users will experience the use of GPS location as extremely privacy invasive.

\subsubsection{Security and Privacy by Implementation}

The rest of the requirements cannot be achieved just with architectural choices, and rather depend on concrete implementation details. We assume that an attacker can potentially collude with or coerce any of the parties in the system (users, location owners, health authority) and impersonate them when interacting with other parties in the system. 
If the adversary coerces all entities in the system, they would have the following capabilities:
\begin{itemize}[topsep=0pt, partopsep=0pt]
\item Attackers can visit locations and obtain their public QR codes.
\item Attackers can modify the QR code of a venue, e.g., by taping a new QR code on top of the old one.
\item Attackers can upload any tracing key (e.g., by colluding with an owner that is authorized to upload)
\item Attackers might corrupt users' phones to read stored information and modify phone behavior.
\item Attackers might monitor network communication between users/venue-owners and backend systems.
\end{itemize}
If the adversary has all of these capabilities, in their widest sense (e.g., visit all locations or continuously corrupt the phone) some requirements cannot be fulfilled. \name does not aim to protect users in those cases, as there is no technological defense possible. Concretely, we do not aim at protecting from the following three cases.

First, the case of a health authority that decides to trigger notifications from a venue, even though no SARS-CoV-2-positive patient has declared having visited that location. We assume the health authority has process in place to protect from such function creep. We also show that, as long as the health authority only starts triggering procedures for places declared by SARS-CoV-2-positive patients, the health authority cannot trigger notifications without the collaboration of the owner.

Second, the case of an attacker that has \emph{continuous} control over a users' phone. Such an adversary can hide notifications resulting in denial of service, show false notifications, keep records of locations visited by users, and even determine whether they have been notified; and it is not possible to protect against these attacks. We only consider attackers with one-off access to the phone when analyzing the privacy of visited locations. We note that even given access to the phone, the adversary cannot learn information about trace locations that the user did not visit, nor information related to positive users.
 
Third, if the adversary controls both the health authority and the QR code at the location (i.e., the adversary knows the secret key of the QR code scanned by the user) we cannot protect the user against notifications. The adversary has all the information to trigger a notification at a place where the user has scanned a QR code. By F2, this user should be notified. In other words, the functional requirements imply that no protection against this attack is possible.

In the following, we detail under which adversarial conditions \name can fulfill which of the remaining requirements.

\para{Preventing False Notifications (S1, S2)} As QR codes are printed and posted in a location, it is difficult to tailor them to specific visitors. Therefore, it is difficult to single out specific users even if all of the rest of the system protections fail, ensuring S1. When it comes to groups (S2), we aim to prevent attackers from triggering false notifications in the following situations:

\begin{itemize}[topsep=1pt, partopsep=0pt]
\item \textit{Trace security.} Users that check-in only to honest venues (i.e., the QR code was generated by an honest owner, and not modified) will not be notified as long as the honest owners do not upload tracing information. This property holds even against attackers that collude with the health authority (assuming that the HA does not compel tracing information from the honest venue).
\item \textit{Info binding.} Users that scan arbitrarily modified QR codes will not be notified as long as the health authority is honest and the locations corresponding to the scan are not marked as trace locations.
\item \textit{Entry binding.} Modifications of honestly generated QR codes never result in notifications, not even when the health authority initiates tracing for the original locations.
\end{itemize}

\para{Preserving privacy of Trace Locations (PL1)} As a result of the functional requirements, health authorities learn which locations are marked as trace locations. Similarly, users that are notified learn (or could deduce) which locations are marked as trace locations. We aim to provide privacy of trace locations against non-visitors of a location. Non-visitors (that do not collude with visitors) cannot learn whether that location has been notified.

\para{Preserving privacy of visited locations (PU3)} Information about visits stored on the phone should not reveal the visited locations to snapshot attackers that gain one-off access to the data stored on a user's phone, i.e., records stored on the phone that do not coincide with the tracing window for a trace location remain unlinkable.
This property even holds when an attacker gains access to the data stored by location owners, as long as the health authority does not cooperate.

\subsection{A fallback solution: pen \& paper}
To satisfy the requirement for a paper-based fallback solution for visitors that do not have or want to use the app (F1), we present a fallback pen \& paper solution. This solution also serves as a baseline for the proposed digital solution.

This approach works as follows:

\begin{enumerate}
\item A location keeps a box/list per day or time-slot.
\item Every visitor enters contact information (e.g., a phone number or an e-mail address) into the box/list.
\item Only when contact tracing reveals that a SARS-CoV-2-positive person visited a location during a specific time: the authorities receive the corresponding list \& contact every presence contact on this list.
\item After 14 days, the boxes/lists are destroyed.
\end{enumerate}

We refer to Section~\ref{sec:evaluation-existing} for an evaluation of this
paper-based fallback approach.

\subsection{Cryptographic Building Blocks}

\name relies on an identity-based encryption scheme to provide its most important properties. In an identity-based encryption scheme, messages can be encrypted against identities without requiring a specific key-pair to be generated for each identity. Instead, a trust authority -- in our case a location owner -- generates a master public key $\masterpk$ and a corresponding master private key $\mastersk$ by running the $\ibekeygen$ algorithm. We emphasize that each location has its own corresponding public key $\masterpk$.

To encrypt a message $m$ against an identity $\id$ under the public key $\masterpk$, a party (in our case a visitor) runs $\ctxt \gets \ibeenc(\masterpk, \id, m)$. To decrypt this ciphertext, the trust authority (e.g., location) first computes the corresponding identity-based decryption key $\skid \gets \ibekeyder(\masterpk, \mastersk, \id)$. Given the identity-based decryption key $\skid$ we can then decrypt a ciphertext $\ctxt$ encrypted under an identity $\id$ by running $m \gets \ibedec(\id,\allowbreak \skid,\allowbreak \ctxt)$.

\name uses use a slight modification of the FullIdent Boneh-Franklin scheme~\cite{BonehF01}. (The only modification is that the randomness $r$ now also depends on the identity $\id$, which is passed to $\ibedec$ for verification purposes.) This CCA2 secure scheme ensures ciphertexts are robust. They can only be decrypted using the correct identity-based decryption key. Thus protecting against fake notifications (S1, S2). Moreover, this scheme has strong anonymity properties. It ensures that a ciphertext does not reveal any information about the identity under which it was created, nor the corresponding master public key. Thus ensuring privacy of records (PU3).

Finally, \name uses a regular public-key encryption scheme given by the algorithms $\keygen$, $\enc$, and $\dec$ with the usual semantics. \name also uses a (symmetric) authenticated encryption scheme given by the algorithms $\aegen,\allowbreak \aeenc, \allowbreak \aedec$. The key generator algorithm $k \gets \aegen(1^\secpar)$ takes as input a security parameter $\secpar$ and outputs a key $k$. The encryption algorithm $c \gets \aeenc(k, m)$ takes as input a key $k$ and a message $m \in \{0, 1\}^*$ and outputs a ciphertext $c$. The decryption algorithm $m \gets \aedec(k, c)$ takes as input a key $k$ and a ciphertext $c$, and outputs a message $m$ or an error symbol $\bot$.

Throughout $\secpar$ denotes the security parameter in bits. In practice, $\secpar = 128$ suffices.

\subsection{Digital presence tracing}

The digital presence tracing solution works as follows. We assume that users have the corresponding app installed on their phone.

\paragraph{Setup} The health authority generates a public-private encryption key pair $\pkH, \skH$ (by running \keygen) and picks a public hash function $\hash$. It also sets up the IBE scheme by running $\pp \gets \ibecommonsetup(1^\secpar)$. The public parameters $\pp$ describe pairing groups
$\group_1, \group_2$, and $\group_T$ of prime order $\grouporder$ generated by generators $\generator_1, \generator_2, \generator_T$ such that there exists a bilinear pairing $\pairsym: \group_1 \times \group_2 \to \group_T$. The parameters also describe $\hash_1 : \{0,1\}^* \to \group_1$, a hash function mapping strings into group elements of $\group_1$. The health authority publishes $\pkH, \pp$, and $\hash$.

\paragraph{QR Code Generation} A location takes as input the name and place (e.g., address) of the location encoded as a string $\info$ and generates a set of QR codes as follows. It computes two IBE key pairs (one for itself, the location, and one for the health authority)
\begin{align*}
  (\masterpkvenue, \masterskvenue) &\gets \ibekeygen(\pp), \\
  (\masterpkhealth, \masterskhealth) &\gets \ibekeygen(\pp)
\end{align*}
and computes $\masterpk = \masterpkvenue \cdot \masterpkhealth \in \group_2$. It encrypts the master secret key $\masterskhealth$ for the health authority by creating the ciphertext $\ctxtha = \enc(\pkH, \masterskhealth)$. Next, it picks random nonces $\nonce_1,\allowbreak \nonce_2 \in \{0, 1\}^{2\secpar}$ and a random symmetric key $\notificationkey \gets \aegen(1^\secpar)$. Let
\begin{align*}
  \entry &= \masterpk,\\
  \entryproof &= (\nonce_1, \nonce_2),\\
  \mastertrace &= (\masterpk, \masterskvenue, \info, \nonce_1,\nonce_2, \ctxtha).
\end{align*}
The values $\entry$ and $\entryproof$ will become part of the entry QR code. The master tracing key $\mastertrace$ is needed to perform any type of presence tracing. Part of the master private key is stored in encrypted form as $\ctxtha$. Cooperation of the health authority is needed to decrypt this value, and to compute tracing keys.

The location constructs two QR codes
\begin{align*}
  \qrentry &= \qrcode(\textsf{``entry''} \parallel \info \parallel \entry \parallel \entryproof \parallel \notificationkey ) \\
  \qrtrace &= \qrcode(\mastertrace \parallel \notificationkey)
\end{align*}
and prints them. The code \qrentry are pasted at the entrance(s) of the location or on individual tables. The code \qrtrace is kept private by the organizer.

The notification key \notificationkey is used to encode messages by the health authority so that they can only be decrypted by apps of users that actually visited the corresponding location (or that collude with somebody that did). This protection is not perfect. Anybody that knows or learns \notificationkey can subsequently make it public.

\parait{Safely generating the QR codes} It is essential that the master private key $\masterpkvenue$ is only known to the location owner. Therefore, these QR codes cannot be generated server-side. We propose to use client-side JavaScript to statelessly generate PDFs containing the QR codes. This PDF should contain two pages:
\begin{itemize}
\item A page that includes the entry QR code \qrentry, with clear instructions labeling this as the entry code, explanations of fall backs, and the name of the location.
\item A page with the tracing QR code \qrtrace.
\end{itemize}

The QR code should encode a URL that opens the app directly, without generating further network traffic.

\paragraph{Entering/leaving a venue} Upon entering a venue, the user uses their app to scan the corresponding entry QR code $\qrentry$.
Let $\info$, $\entry = \masterpk$, $\entryproof = (\nonce_1, \nonce_2)$, and $\notificationkey$ be the values contained therein. Here $\info$ contains the name and address of the location. The app shows $\info$ to the user and asks for confirmation that the user wants to check in.

\newcommand{\hourctr}{\code{hourctr}}
Once the user confirms a checkout (e.g. on their own initiative, or after a reminder from the app) the app proceeds as follows. Let the message
\begin{equation*}
  m = (\texttt{arrival time},\enspace \texttt{departure time},\enspace \notificationkey)
\end{equation*}
contain the entry and departure time, and the notification key $\notificationkey$. For each hour overlapping with the user's presence at the location, the app computes $\hourctr$, the number of hours since UNIX epoch, and proceeds as follows.
\begin{enumerate}
\item The app computes the identity corresponding to this hour
  \begin{equation*}
    \id = \hash( \hash(\info \parallel \nonce_1) \parallel \hourctr \parallel \nonce_2),
  \end{equation*}
 \item The app computes the ciphertext $\ctxt \gets \ibeenc(\masterpk, \id, m)$ and stores it together with a label for the current day.
\end{enumerate}

Devices automatically delete any entry older than 10 days.

\paragraph{Initiating contact tracing}
After the contact tracing team of the local health authority has determined that a SARS-CoV-2-positive person has visited a location during the contagious period, they proceed as follows. Let \entryplus and \exitplus the times that the SARS-CoV-2-positive person entered and exited this location.
\begin{enumerate}
\item The contact tracing team creates a case number to manage locations related to this patient.
\item They then contact the owner/organiser of the location and request upload of the hour-specific pre tracing keys to the health authority's servers. The location owner scans the tracing QR code $\qrtrace$ containing $\mastertrace = (\masterpk, \masterskvenue, \info, \nonce_1,\nonce_2, \ctxtha)$ and the notification key $\notificationkey$ and proceeds as follows:
  \begin{enumerate}
  \item For each hour $\hourctr$ (in hours since UNIX epoch) overlapping with the time between \entryplus and \exitplus the owner computes the corresponding identity
    \begin{equation*}
      \id_{\hourctr} = \hash( \hash(\info \parallel \nonce_1) \parallel \hourctr \parallel \nonce_2),
    \end{equation*}
    and the partial identity-based decryption key
    \begin{equation*}
      \preskidvenueh = \ibekeyder(\masterskvenue, \id_{\hourctr}).
    \end{equation*}
    Let the pre-tracing key be $\pretrace_{\hourctr} = (\id_{\hourctr}, \preskidvenueh)$.
    \item The owner uploads the case code, $\traceproof = (\masterpk,\allowbreak \nonce_1,\allowbreak \nonce_2, \ctxtha)$, all pre-tracing keys $\preskidvenueh$, and the notification key $\notificationkey$ to the health authority's server.
  \end{enumerate}
\item The health authority's system processes each upload as follows.
  \begin{enumerate}
  \item It decrypts $\ctxtha$ to obtain $\masterskhealth$.
  \item For each hour $\hourctr$ it parses $\pretrace_{\hourctr}$ as $(\id_{\hourctr}, \preskidvenueh)$. And it computes its part of the identity-based decryption key
    \begin{equation*}
      \preskidhealthh = \ibekeyder(\masterskhealth, \id_{\hourctr}).
    \end{equation*}
    and computes the final identity-based decryption key
    \begin{equation*}
    \skidh = \preskidvenueh \cdot \preskidhealth.
    \end{equation*}
    Let $\traceh = (\id, \skidh)$.
  \item It validates the computed tracing key $\traceh = (\id_h, \skidh)$ by first checking that
    \begin{equation*}
      \id_{\hourctr} = \hash( \hash(\info \parallel \nonce_1) \parallel \hourctr \parallel \nonce_2).
    \end{equation*}
    Next, it picks a random message $m$ of sufficient length and computes the ciphertext $\ctxt \gets \ibeenc(\masterpk, \id_{\hourctr}, m)$, and verifies that $\ibedec(\skidh, \id_{\hourctr}, \ctxt) = m$.
    If any check fails, it removes the upload. 
  \end{enumerate}
\item The contact tracing team checks that all uploaded information from the tracing codes (in particular, the description of the location contained in \info) corresponds to the locations the SARS-CoV-2-positive person visited. They remove any extra uploads under this case code.
\item The contact tracing team specify a message $m$ that should be sent to the notified users.
\item Finally, the health authority server computes $\ctxtmsg = \aeenc(\notificationkey,\allowbreak m)$, the encrypted message for the notified visitors, and for each relevant hour $\hourctr$ it publishes a record $(\traceh, \entryplus, \exitplus, \ctxtmsg).$
\end{enumerate}

The checks performed in step 3 ensure that the tracing keys $\traceh$ are correct and correspond to the location specified in $\info$. The check in step 4 ensures that this location is what the health authority requested. Together, these steps ensure that a malicious or coerced venue owner cannot upload tracing information related to another location without this being detectable by the health authority.

\paragraph{Presence tracing and notification}
The user’s app regularly (say, every few hours) performs the following checks:
\begin{enumerate}
\item The app downloads all $(\traceh, \entryprime, \exitprime, \ctxtmsg)$ tuples that were published since the last time it checked.
\item For each tuple downloaded $(\traceh, \entryprime, \exitprime, \ctxtmsg)$ let 
  $\traceh = (\id_{\hourctr}, \skidh)$. The app proceeds as follows.
  \begin{enumerate}
  \item For each record $\ctxt$ recorded on the days included between $\entryprime$ and $\exitprime$, it tries to decrypt it using $\ibedec(\id_{\hourctr},\allowbreak \skidh,\allowbreak \ctxt)$. The app selects records where decryption succeeds (i.e., those not equal to $\bot$).
  \item For all selected records $\ctxt$, the app uses the plaintext of $\ctxt$ to recover the arrival time, departure time and the notification key $\notificationkey$. If there is an overlap between the user's time at the location and what is indicated by $\entryprime$ and $\exitprime$, the app uses $\notificationkey$ to decrypt $\ctxtmsg$ and notifies the user by displaying this message.
  \end{enumerate}
\end{enumerate}
  
\subsubsection{Revisiting requirements}
With the extra detail about the \name schemes provided in the previous sections, we revisit the open questions regarding the requirements. We refer to Section~\ref{sec:sec-priv-design} for an overview.

\paragraph{Functional requirements}
The digital construction in combination with the paper-based fallback solution ensures F1. The digital system keeps track of accurate entry and exit times. Therefore, everybody that potentially overlaps with the SARS-CoV-2-positive user at a location can be notified. Satisfying F2. Moreover, if users accurately enter departure times, users will not be notified if they did not overlap with SARS-CoV-2-positive user. Largely satisfying F3.

\paragraph{Security}
We revisit requirement S2 that it should be difficult to target groups of people. At a high-level, these are satisfied because both the location owner and health authority need to participate to create tracing keys. And because users verify the location they check into, and the health authority verifies the location for which keys are generated, substitution attacks fail. More specifically, in Section~\ref{sec:sec-priv-design} we identified three technical requirements: trace security, info binding, and entry binding, needed to satisfy S2. We briefly discuss each.

The security of the underlying identity-based encryption (IBE) scheme, ensures that the location's master secret key is needed to compute tracing keys. As long as the location owner is honest, this master secret key is not available to the attacker. Satisfying \emph{trace security}.

The robustness of the IBE scheme ensures that decryption of a ciphertext, i.e., record, fails if the supplied identity $\id$ differs from the one used during encryption. Since the app binds $\id$ during encryption to the name and address of the location (contained in $\info$), the user will not receive notifications, as long as the health authority does not accept an upload in steps 3 and 4 of ``Initiating contact tracing''. Any other upload (e.g., for another venue) even with maliciously generated values, will not result in a match. Satisfying \emph{info binding}.

Similarly, any modifications of honestly generated QR codes by other venues do not result in notifications. The user uses all provided values to compute the identities $\id$. So any modification results in different identities. And per robustness of the IBE scheme, ciphertexts encrypted under different identities fail to decrypt. Satisfying \emph{entry binding}.

\paragraph{Location privacy (PL1)}
We argue why an adversary that did not visit a location (nor colluded with somebody who did) cannot learn which locations are trace locations. The tracing records $(\traceh, \entryprime, \exitprime, \ctxtmsg)$ are public. But, by assumption the
the adversary does not know the master public key $\masterpk$ of the location nor the notification key $\notificationkey$.
Since it does not know $\notificationkey$, it gains no information from $\ctxtmsg$. The time slot also does not help identity the location.

What is left is the value $\traceh = (\id_{\hourctr}, \skidh)$. The specific choice of IBE scheme and the preimage resistance of $\hash$ ensures that these values do not help distinguish locations to attackers that did not see $\qrtrace$. Thus PL1 is satisfied.

\paragraph{User privacy (PU3)}
We revisit the user privacy requirement that information stored by the app does not give any information about locations visited by the user. Note the app only stores the IBE ciphertext $\ctxt$ and the day. Anonymity of the IBE scheme guarantees that $\ctxt$ does not reveal any information about the corresponding master public key (of the location) nor the identity $\id$. Thus ensuring PU3.
